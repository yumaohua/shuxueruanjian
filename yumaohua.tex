% !TeX spellcheck = en_US
\documentclass{article}

\usepackage{graphicx}
\usepackage{amsmath}

\title{The First Homework:The Proof of The Inevitability of The Similar Diagonalization of Real Symmetric Matrices in Real Number Field}


\author{Yu Maohua\\Information and Computing Sciences\\3210104374}

\begin{document}

\maketitle

This is a question from the field of algebra.

\section{Question description}

The description of the question is as follows:please prove a real symmetric matrix can be similarly diagonalized in real number field. 
\section{Proof}
lemma 1:Eigenvalues of real symmetric matrices are all real numbers.

proof of lemma 1 Let A be an $n$$\times n $matrix in real number field, $A^T=A$, $\lambda$ is an eigenvalue of $A$, then we have

\begin{equation}
\bar{\xi}^TA\xi = \lambda \bar{\xi}^T \xi =\lambda \left|\xi\right|^2	
\label{1}
\end{equation}

\begin{equation}
\bar{\xi}^TA\xi=(\bar{A\xi})^T\xi=(\bar{\lambda\xi})^T\xi=\bar{\lambda}\left|\xi\right|^2
\label{2}
\end{equation}
with equasion \ref{1} and \ref{2} we have ($\lambda$-$\bar{\lambda}$)$\left|\xi\right|^2$=0,\\
so $\lambda$ is a real number, and eigenvectors of A are in $R^n$.
\\
\\
lemma 2:Eigenvectors belonging to different eigenvalues of a real symmetric matrix are orthogonal.

proof of lemma 2 Let $\lambda_1$ and $\lambda_2$ be different eigenvalues of real symmetric matrix A, and A$\xi_1$=$\lambda_1$$\xi_1$,A$\xi_2$=$\lambda_2$$\xi_2$\\
so~~~~~~~~~~~~~~~~~~~~~~~~~~~~~$\lambda_1$$(\xi_1,\xi_2)$=$\lambda_2$$(\xi_1,\xi_2)$.\\
Because $\lambda_1$ is different from $\lambda_2$, ($\xi_1$,$\xi_2$)=0.\\
\section{proof of theorem}
We will use induction.
Let A be an $n$$\times$$n$matrix in real number field, we just need to prove there is a unit orthogonal matrix U makes $U^TAU$ a diagonal matrix.\\
For n=1, U={1};\\
for n=k-1, we assume $U_0$ makes $U_0^TAU_0$ a diagonal matrix.\\
For n=k,with lemma 1,we know that A only has real eigenvalues $\lambda_1$ to $\lambda_n$. The eigenvector of $\lambda_1$ is $\xi_1$. With Schmidt orthogonal, we have Sdandard orthogonal basis $\xi_1$ to $\xi_n$. They make a unit orthogonal matrix $U_1$. $U_1^TAU_1=X$;X is a quasi upper triangular matrix.Then we have proved it with assumption.
























\end{document}
