\documentclass{ctexart}

\usepackage{graphicx}
\usepackage{amsmath}

\title{作业三:规划我的Linux工作环境}

\author{俞茂桦\\信息与计算科学\\3210104374}

\begin{document}
	
	\maketitle
	
	\section{我的Linux发行版名称以及版本号}
	
	我的Linux版本信息如下:
	\begin{verbatim}
		LSB Version:	core-9.20170808ubuntu1-noarch:security-9.20170808ubuntu1-noarch
		Distributor ID:	Ubuntu
		Description:	Ubuntu 18.04.6 LTS
		Release:	18.04
		Codename:	bionic
		
	\end{verbatim}
	\section{我的调整、软件安装与额外配置}
	为了适应专业学习的需要,我对我的系统做了一些调整,安装了需要的软件,并做了一些额外配置。
	\subsection{我对系统的主要调整}
	\begin{flushleft}
	1.在\verb|Software & Updates|中更改下载源为\verb|https://mirrors.cn99.com/ubuntu|,加快我的下载速度。\\
	2.在\verb|Language Support|中将\verb|Keyboard input method system|改为\verb|fcitx|,配合\verb|google pinyin|进行中文输入。\\
	3.安装了增强功能。\\
	4.设置了虚拟机的输入法,显存,共享文件夹等。
	\end{flushleft}
	\subsection{我安装的包}
	\begin{flushleft}
		1.synaptic package manager:为一个包管理器,可以帮助我下载软件\\
		2.make \verb|&| cmake \verb|&| automake:帮助项目完成编译\\
		3.emacs \verb|&| emacs-goodies-el:著名的集成开发环境和文本编辑器\\
		4.texlive-full:写相关报告使用,适应数学报告及论文的需求\\
		5.doxygen \verb|&| doxygen-doc:将程序中的特定注释转换为说明文件\\
		6.libboost-all-dev:数值计算相关的C++代码\\
		7.trilinos-all-dev \verb|&| doc \verb|&| dbg:并行计算相关\\
		8.dx:后处理相关画画软件\\
		9.ssh:字符界面远程登录相关\\
		10.vnc4server \verb|&| x11vnc:图形界面远程登录相关\\
		11.git:源代码管理器\\
		12.texstudio:编译tex文本环境\\
	\end{flushleft}
	\subsection{额外的配置工作}
	\begin{flushleft}
		1.将emacs可输入中文并打开emacs指令简化为emacs\\
		2.配置了emacs环境\\
		3.更改了texstudio默认编译器,使其可编译中文\\
	\end{flushleft}
	\section{我下一步的工作}
	\subsection{对未来使用Linux环境的场合的预计}
	\begin{flushleft}
		1.专业课程如“数据结构与算法”等课程上学习报告的书写将在Linux环境下使用\LaTeX{}文本,相关程序将会在Linux环境下编写和运行\\
		2.专业方向诸如数值计算、并行计算以及其它内容将在Linux环境下进行,因为在Linux环境下可以获得开源的合适的算法\\
		3.毕业论文的完成\\
		4.据分析,信息与计算科学专业在工程领域可以完成数学计算、科学计算的工作,在体育领域可以完成临场数据统计与处理等工作,在金融领域可以完成证劵模型计算等工作,不论在什么领域面对的都是计算与软件开发相关内容,因此需要Linux这样开源且高效的工作环境\cite{高泽健2020浅析信息与计算科学在几个领域中的应用}\\
	\end{flushleft}
	\subsection{对目前工作环境的评估}
	几经波折,\LaTeX{}的编写环境已经非常使我满意,但是在emacs中我目前只配置了输入中文和语法高亮,还没有办法自动补齐,对于未来编写代码有些不方便。为防止现有系统被破坏,我正尝试在另一台虚拟机中将emacs配置齐全。
	\section{我如何保证工作系统中代码、文献和工作结果的安全}
	\begin{flushleft}
		1.将二进制文件和文档分开\\
		2.及时将自己的文档上传至github\\
		3.将pdf等二进制文件上传至坚果云进行保存\\
		4.在Linux恶意软件增加的环境下,对软件安装保持谨慎\cite{崔丹1}\\
		
	\end{flushleft}
	
\bibliographystyle{unsrt}
\bibliography{learn}

\end{document}